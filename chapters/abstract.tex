%\vspace*{\fill}

\vspace*{2cm}
\begin{addmargin}[1em]{1em}% 1em left, 1em right
\hspace{8mm}
\begin{center}
\subsection*{Abstract}
\end{center}

%\blindtext

Mesh protocols have been providing connectivity to smart homes and
wireless sensing for about two decades. There is a variety of approaches to
building mesh networks proposed by different research groups. Two protocols: Zigbee 
and Thread are discussed and compared in this thesis. Although they have similar features
and behaviors, there are some significant differences in the architectures of
networks built with them.

Comparing the performance of the two mentioned protocols running nRF SoCc has been a problem because
method available for users of these devices had not existed.
This dissertation provides an overview of these protocols alongside the
performance evaluation of networks built with them.
Moreover, the method of
collecting performance metrics of Zigbee and Thread protocols is proposed.
This method consists of using applications built with Nordic Semiconductor Connect 
SDK. It was used to perform the latency and throughput measurements of networks 
with up to 7 nodes. In a summary, the results are presented and discussed. It
is concluded that Thread introduces lower latency and higher throughput than 
Zigbee. Although the results show Thread's advantage over Zigbee, there
are other factors than performance that are important when a network protocol
is chosen for a product.

\end{addmargin}
\vspace{2cm}
\clearpage