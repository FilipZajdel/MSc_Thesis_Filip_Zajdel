
\section{Conclusion of performance comparison between Thread and Zigbee}

Both Zigbee and Thread are popular standards for various applications.
Although performance comparison showed that Thread has lower latency and higher
throughput when compared to Zigbee,
there are other important factors that need to be taken into consideration
while choosing a technology for the product. Zigbee has been in the 
market for the longer period than Thread and has been adopted by many
vendors. On the other hand, Thread enables for easier integration
with Internet, which can be an asset for some applications.

Both protocols require little computing power which is good in terms
of energy consumption. However, Zigbee required less RAM than Thread.

To sum up, a technology choice should not be taken based just on
the performance. The purposes of the two compared protocols are slightly
different. When designing a product, one must choose the protocol that
matches the product specification and enables its further development and
certification.

\section{Conclusion of research methodology}

The proposed methodology relies on testing applications which exchange
a packets stream between chosen nodes in the network. Both programs
compute the packets delay, throughput and some other statistics which
were not used in the experiment.

Among other objectives, this thesis aimed to develop such a methodology
which could be used in the future for testing new versions of protocols.
Moreover, the NCS framework hasn't included such a programs before and 
it can be a good opportunity for Nordic Semiconductor's customers to
test the provided technology on their own.

\section{Future work}

Keeping
the software up to date with constantly changing libraries is
challenging and that is one of the goals for the future. Moreover,
the developed methodology should be shared to broader community
and tested against setup in the environment, where mesh networks
are usually located.
