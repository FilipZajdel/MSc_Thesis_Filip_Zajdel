%\vspace*{\fill}

\vspace*{2cm}
\begin{addmargin}[1em]{1em}% 1em left, 1em right
\hspace{8mm}
\begin{center}
\subsection*{Oświadczenie studenta}
\end{center}
Uprzedzony(-a) o odpowiedzialności karnej na podstawie art. 115 ust. 1 i 2 ustawy z dnia 4 lutego 1994 r. o prawie autorskim i prawach pokrewnych (t.j. Dz.U. z 2018 r. poz. 1191 z późn. zm.): „Kto przywłaszcza sobie autorstwo albo wprowadza w bł\k{a}d co do autorstwa całości lub części cudzego utworu albo artystycznego wykonania, podlega grzywnie, karze ograniczenia wolności albo pozbawienia wolności do lat 3. Tej samej karze podlega, kto rozpowszechnia bez podania nazwiska lub pseudonimu twórcy cudzy utwór w wersji oryginalnej albo w postaci opracowania, artystyczne wykonanie albo publicznie zniekształca taki utwór, artystyczne wykonanie, fonogram, wideogram lub nadanie.”, a także uprzedzony(-a) o odpowiedzialności dyscyplinarnej na podstawie art. 307 ust. 1 ustawy z dnia 20 lipca 2018 r. Prawo o szkolnictwie wyższym i nauce (Dz. U. z 2018 r. poz. 1668 z późn. zm.) „Student podlega odpowiedzialności dyscyplinarnej za naruszenie przepisów obowi\k{a}zuj\k{a}cych w uczelni oraz za czyn uchybiaj\k{a}cy godności studenta.”, oświadczam, że niniejsz\k{a} pracę dyplomow\k{a} wykonałem(-am) osobiście i samodzielnie i nie korzystałem(-am) ze źródeł innych niż wymienione w pracy. 

\medskip
Jednocześnie Uczelnia informuje, że zgodnie z art. 15a ww. ustawy o prawie autorskim i prawach pokrewnych Uczelni przysługuje pierwszeństwo w opublikowaniu pracy dyplomowej studenta. Jeżeli Uczelnia nie opublikowała pracy dyplomowej w terminie 6 miesięcy od dnia jej obrony, autor może j\k{a} opublikować, chyba że praca jest części\k{a} utworu zbiorowego. Ponadto Uczelnia jako podmiot, o którym mowa w art. 7 ust. 1 pkt 1 ustawy z dnia 20 lipca 2018 r. – Prawo o szkolnictwie wyższym i nauce (Dz. U. z 2018 r. poz. 1668 z późn. zm.), może korzystać bez wynagrodzenia i bez konieczności uzyskania zgody autora z utworu stworzonego przez studenta w wyniku wykonywania obowi\k{a}zków zwi\k{a}zanych z odbywaniem studiów, udostępniać utwór ministrowi właściwemu do spraw szkolnictwa wyższego i nauki oraz korzystać z utworów znajduj\k{a}cych się w prowadzonych przez niego bazach danych, w celu sprawdzania z wykorzystaniem systemu antyplagiatowego. Minister właściwy do spraw szkolnictwa wyższego i nauki może korzystać z prac dyplomowych znajduj\k{a}cych się w prowadzonych przez niego bazach danych w zakresie niezbędnym do zapewnienia prawidłowego utrzymania i rozwoju tych baz oraz współpracuj\k{a}cych z nimi systemów informatycznych. 

\end{addmargin}
\vspace{2cm}

\begin{addmargin}{1em}
\begin{flushright}
\makebox[6cm][s]{\dotfill}\par
\makebox[6cm][c]{\small (czytelny podpis studenta)}
\end{flushright}
\end{addmargin}