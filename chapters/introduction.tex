
\section{Motivation}

An expanding market of IoT devices requires manufacturers and their engineers to adopt to new standards or rework 
the old ones. The enterprise IoT market has grown by 23\% since 2021 \cite{IoTMarketSize}. A huge part of this business is taken
by home automation products which wouldn't exist without chips combining power saving designs and wireless connectivity.

This dissertation focuses on two network protocols used mainly by home automation devices: Zigbee and Thread.
Both of them are supported by Nordic Semiconductor's SoC nRF52840. Although these protocols share many features
and are based on the same physical layer- IEEE 802.15.4, their designs and targets are different.

\section{Problem statement}

Zigbee as well Thread are standards designed for energy efficient low-data-rate applications. Both protocols share the same purpose and are 
built upon the same MAC layer. Although mesh networks using them behave 
similarly, their protocol stacks significantly differ.

The main focus of this thesis is to develop methods of measuring the
performance of these protocols and compare them in terms of gathered results.

\section{Objectives}

The main objective is to develop the methodology for performance evaluation 
of Zigbee and Thread run on Nordic Semiconductor's SoCs. These measures
aim to gather performance metrics, which are then used for comparison 
of these protocols.

\section{Contribution}

The contribution of this thesis include study of Thread and Zigbee 
protocols, porting and modifying benchmark and zperf application for
performance measurement and results analysis.
